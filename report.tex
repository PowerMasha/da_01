\documentclass[12pt]{article}

\usepackage{fullpage}
\usepackage{multicol,multirow}
\usepackage{tabularx}
\usepackage{listings}
\usepackage{color}
\usepackage[table]{xcolor}
\usepackage{tikz}
\usepackage{ulem}
\usepackage[russian]{babel}
\usepackage[utf8x]{inputenc}
\usepackage[T1, T2A]{fontenc}
\usepackage{fullpage}
\usepackage{multicol,multirow}
\usepackage{tabularx}
\usepackage{ulem}
\usepackage{tikz}
\usepackage{pgfplots}
\usepackage{indentfirst}

% Оригиналный шаблон: http://k806.ru/dalabs/da-report-template-2012.tex

\begin{document}

\section*{Лабораторная работа №\,1 по курсу дискрeтного анализа: сортировка за линейное время}

Выполнил студент группы М80-208Б-18 МАИ \textit{Алексеева Мария}.

\subsection*{Условие}
\begin{enumerate}
\item Требуется разработать программу, осуществляющую ввод пар «ключ-значение», их упорядочивание по возрастанию ключа сортировкой подсчетом и вывод отсортированной последовательности. 
\item Вариант 2-2.

Сортировка подсчетом.

Тип ключа: почтовые индексы.

Тип значения: строка до 2048 символов.
\end{enumerate}

\subsection*{Метод решения}

\begin{enumerate}
    \item Данные на вход программе подаются через стандартный поток ввода,   
    \begin{lstlisting}[language=C++]
    std::istream &operator>>(std::istream &in, TKeyValue &m) {
	in >> m.key >> m.value;
	return in;
    }
    \end{lstlisting}
    \item Так как количесво данных заранее не известно, необходимо будет реализовать вектор (по условию задания стандартный вектор из библиотеки STL использовать запрещается).
    \item Ключи будут представляться в виде беззнакового числа и в процессе ввода будет видоизменять число, заполняя его нулями при необходимости.      out << std::setw(6) << std::setfill('0') << m.key;.
    \item Сортировка подсчётом принимает тот же вектор входных данных и максимальное значение ключа, а также создаёт 2 новых массива: \textbf{b}, в который будет записана отсортированная последовательность, и \textbf{c}, в котором подсчитывается число повторений каждого возможного символа в сортируемом разряде ключа.
\end{enumerate}

\subsection*{Описание программы}

На каждой непустой строке входного файла располагается пара «ключ-значение»,
поэтому создадим новую структуру \textbf{TKeyValue}, в которой будем хранить ключ и значение.

\begin{table}[!htb]
\begin{tabular}{|m{8cm}|m{8cm}|}
\hline
\multicolumn{2}{|c|}{main.cpp} \\ 
\hline
\cellcolor{gray!25} Тип данных       & \cellcolor{gray!25} Значение\\ 
\hline
struct TKeyValue & Структура для хранения пары "ключ-значение" \\ 
\hline
class TVector & Вектор для хранения структур \textbf{TData}\\
\hline
\cellcolor{gray!25} Функция & \cellcolor{gray!25}Значение\\
\hline
TVector() & Конструктор класса \textbf{TVector} \\
\hline
TVector(size_t) & Конструктор класса \textbf{TVector} \\
\hline
$\sim$ TVector() & Деструктор класса \textbf{TVector} \\
\hline
void PushBack(TData) & Добавить элемент в конец вектора\\
\hline
void CountingSort(size\_t) & Функция сортировки подсчётом\\
\hline
int main() & Главная функция, в которой происходит чтение данных, вызов функции сортировки и вывод. \\
\hline
\end{tabular}
\end{table}

\newpage

\subsection*{Дневник отладки}

При создании этой таблицы была использована история посылок.
\begin{table}[!htb]
\begin{tabular}{|m{2cm}|m{3cm}|m{10cm}|}

\hline
\cellcolor{gray!25} время       & \cellcolor{gray!25} ошибка & \cellcolor{gray!25} причина\\ 
\hline
2019/11/30 15:15:00 & Превышено время & В главной функции создавался новый результирующий вектор, который копировал отсортированный, что замедляло время работы программы.\\ 
\hline
2019/11/30 15:16:49	& Превышено реальное время работы &  Проблема в неоптимизированном выводе. Использованные методы оптимизации: 1) Опция std::ios_base::sync_with_stdio(false), которая отключает синхронизацию потоков Си и С++, т. е. данные перестат копироваться из буфера в буфер. 2) Опция std::cin.tie(nullptr). Она отделяет изначально связанные потоки cin и cout. При их связи один из потоков сбрасывается перед каждой операцией в другом потоке.\\
\hline
2019/12/05 12:04:49	 & Превышено реальное время работы &   Изменено редактирование ключа с помощью функций std::setw(6) и std::setfill('0'). До этого использовался массив из 6-ти элементов. \\
\hline
2019/12/05 12:16:07	 & Работает & Далее убирались комментарии и лишние функции, используемые только для отладки программы, вводились незначительные косметические изменения. \\
\hline
\end{tabular}
\end{table}

\subsection*{Тест производительности}

Для генерации тестов использовалась следующая программа:
\begin{lstlisting}[language=C++]
#include <iostream>
#include <fstream>
#include <ctime>
#include <cmath>
#include <cstdlib>

int main(int argc, char *argv[]) {
    std::ofstream file(argv[1]);
    srand(time(0));
    size_t size = atoi(argv[2]);
    char key[6];
    
    for (int i = 0; i < size; i++) {
        for (int j = 0; j < 6; j++) {
            key[j] = (char) (rand() % 10 + 48);
        }
        
	int length = rand() % 2050;
        char data[length];
        for (int k = 0; k < length-1; k++) {
            data[k] = (char) (rand() % 26 + 97);
        }
	data[length-1] = '\0';
	file << key << '\t' << data << '\n';
    }
    return 0;
}
\end{lstlisting}

\begin{tikzpicture}
	\begin{axis}[ylabel=Время,xlabel=Количество строк, width=15.5cm, height=10cm,grid=both]
	\addplot coordinates {
		( 100000, 0.625 )
		( 90000, 0.585 )
		( 80000, 0.548 )
		( 70000, 0.421 )
		( 60000, 0.359 )
		( 50000, 0.267 )
		( 40000, 0.207 )
		( 30000, 0.201  )
		( 20000, 0.161  )
		( 10000, 0.082  )};
	\addplot coordinates {
	( 10000, 0.082)
	( 100000, 0.625 )};
	\end{axis}

\end{tikzpicture}


\subsection*{Выводы}

Данную программу можно использовать для сортировки файлов по ключу, который представляет собой постовый индекс.

В процессе работы я получила ценный опыт отладки программы и её оптимизации при помощи утилиты valgrind. Также разобралась в работе сортировки подсчетом, научилась различать устойчивые сортировки от неустойчивых.
\end{document}

